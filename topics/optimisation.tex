% Emacs, this is -*-latex-*

\ex{Gradient of vector-valued functions}
\label{ex:grad-vector}
For a function $J$ that maps a column vector $\w \in \mathbb{R}^n$ to $\mathbb{R}$, the gradient is defined as
\begin{equation}
  \nabla J(\w) = \left(
    \begin{array}{c}
      \frac{\partial J(\w)}{\partial w_1} \\ %
      \vdots \\
      \frac{\partial J(\w)}{\partial w_n}
    \end{array} \right),
\end{equation}
where $\partial J(\w)/\partial w_i$ are the partial derivatives of $J(\w)$ with
respect to the $i$-th element of the vector $\w = (w_1, \ldots, w_n)^\top$ (in
the standard basis). Alternatively, it is defined to be the column vector $\nabla J(\w)$ such that
\begin{align}
  J(\w+\epsilon \h) &=  J(\w) + \epsilon \left( \nabla J(\w)\right)^\top \h+O(\epsilon^2) \label{eq:grad-vector}
\end{align}
for an arbitrary perturbation $\epsilon \h$. This phrases the derivative in
terms of a first-order, or affine, approximation to the perturbed function
$J(\w+\epsilon \h)$. The derivative $\nabla J$ is a linear transformation that
maps $\h \in \mathbb{R}^n$ to $\mathbb{R}$ \citep[see e.g.][Chapter 9, for a
formal treatment of derivatives]{Rudin1976}.

Use either definition to determine $\nabla J(\w)$ for the following functions where $\a \in \mathbb{R}^n$, $\A \in \mathbb{R}^{n \times n}$ and
$f: \mathbb{R} \rightarrow \mathbb{R}$ is a differentiable function.
\begin{exenumerate}
\item $J(\w)=\a^\top \w$.

  \begin{solution}
    First method:
    \begin{align}
      J(\w) &= \a^\top\w = \sum_{k=1}^n a_kw_k& \implies&& \frac{\partial{J(\w)}}{\partial{w_i}} &= a_i
    \end{align}
    Hence
    \begin{equation}
      \nabla J(\w) = \begin{pmatrix} a_1 \\ a_2 \\ \vdots \\ a_n \end{pmatrix} = \a.
    \end{equation}
    Second method:
    \begin{align}
      J(\w + \epsilon\h) &= \a^\top(\w + \epsilon\h) = \underbrace{\a^\top\w}_{J(\w)} + \epsilon \underbrace{\a^\top \h}_{\nabla J^\top\h}
    \end{align}
    Hence we find again $\nabla J(\w) = \a$.
  \end{solution}

\item $J(\w)=\w^\top \A\w$.

  \begin{solution}
    First method: We start with
    \begin{align}
      J(\w) &= \w^\top\A\w = \sum_{i=1}^n \sum_{j=1}^n w_iA_{ij}w_j
    \end{align}
    Hence,
    \begin{align}
      \frac{\partial{J(\w)}}{\partial{w_k}} &= \sum_{j=1}^n A_{kj}w_j + \sum_{i=1}^n w_i A_{ik} \\
                                            &= \sum_{j=1}^n A_{kj}w_j + \sum_{i=1}^n w_i (\A^\top)_{ki}\\
                                            &= \sum_{j=1}^m \left(A_{kj}+(\A^\top)_{kj}\right)w_j
    \end{align}
    where we have used that the entry in row $i$ and column $k$ of the matrix $\A$ equals the entry in row $k$ and column $i$ of its transpose $\A^\top$. It follows that
    \begin{align}
      \nabla J(\w) &=
                     \begin{pmatrix}
                       \sum_{j=1}^n  \left(A_{1j}+(\A^\top)_{1j}\right)w_j\\
                       \vdots \\
                       \sum_{j=1}^n  \left(A_{nj}+(\A^\top)_{nj}\right)w_j
                     \end{pmatrix} \\
                   &= (\A+\A^\top)\w,
    \end{align}
    where we have used that sums like $\sum_j B_{ij} w_j$ are equal to the $i$-th element of the matrix-vector product $\B \w$.

    Second method:
    \begin{align}
      J(\w + \epsilon\h) &= (\w + \epsilon\h)^\top \A(\w + \epsilon\h) \\
                         &= \w^\top\A\w + \w^\top\A(\epsilon\h) + \epsilon\h^\top\A\w +
                           \underbrace{\epsilon\h^\top \A\epsilon\h}_{O(\epsilon^2)} \\
                         &= \w^\top\A\w + \epsilon(\w^\top\A\h + \w^\top\A^\top\h) + O(\epsilon^2) \\
                         &= \underbrace{\w^\top\A\w}_{J(\w)} +\epsilon(\underbrace{\w^\top\A + \w^\top\A^\top}_{\nabla J(\w)^\top})\h + O(\epsilon^2)
    \end{align}
    where we have used that $\h^\top\A\w$ is a scalar so that $\h^\top\A\w = (\h^\top\A\w)^\top = \w^\top \A^\top \h$. Hence
    \begin{equation}
      \nabla J(\w)^\top = \w^\top\A + \w^\top\A^\top = \w^\top(\A+\A^\top)
    \end{equation}
    and
    \begin{equation}
      \nabla J(\w) = (\A + \A^\top)\w.
    \end{equation}
  \end{solution}

\item $J(\w)=\w^\top \w$.

  \begin{solution}
    The easiest way to calculate the gradient of $J(\w) = \w^\top\w$ is to use the previous question with $\A = \I$ (the identity matrix).
    Therefore
    \begin{equation}
      \nabla J(\w) = \I\w + \I^\top\w = \w + \w = 2\w.
    \end{equation}

  \end{solution}

\item $J(\w)=||\w||_2$.

  \begin{solution}
    Note that $||\w||_2 = \sqrt{\w^\top \w}$.

    First method: We use the chain rule
    \begin{equation}
      \frac{\partial{J(\w)}}{\partial{w_k}} = \frac{\partial \sqrt{\w^\top \w}}{\partial \w^\top \w}\frac{\partial \w^\top \w}{\partial w_k}
    \end{equation}
    and that
    \begin{equation}
      \frac{\partial \sqrt{\w^\top \w}}{\partial \w^\top \w} = \frac{1}{2\sqrt{\w^\top \w}}
    \end{equation}
    The derivatives $\partial \w^\top \w/\partial w_k$ were calculated in the question above so that
    \begin{equation}
      \nabla J(\w) = \frac{1}{2\sqrt{\w^\top \w}} 2 \w = \frac{\w}{||\w||_2}
    \end{equation}
    Second method: Let $f(\w) = \w^\top \w$. From the previous question, we know that
    \begin{align}
      f(\w + \epsilon \h) & = f(\w) + \epsilon 2\w^\top \h + O(\epsilon^2).
    \end{align}
    Moreover,
    \begin{align}
      \sqrt{z + \epsilon u + O(\epsilon^2)} & = \sqrt{z} + \frac{1}{2\sqrt{z}}(\epsilon u + O(\epsilon^2)) + O(\epsilon^2)\\
                                            & = \sqrt{z} + \epsilon \frac{1}{2\sqrt{z}} u + O(\epsilon^2)
    \end{align}
    With $z=f(\w)$ and $u = 2 \w^\top \h$, we thus obtain
    \begin{align}
      J(\w + \epsilon \h) &= \sqrt{f(\w+\epsilon \h)} \\
                          &= \sqrt{f(\w)} + \epsilon \frac{1}{2\sqrt{f(\w)}} 2\w^\top \h + O(\epsilon^2)\\
                          &= \sqrt{f(\w)} + \epsilon \frac{\w^\top}{\sqrt{f(\w)}}\h + O(\epsilon^2)\\
                          &= J(\w) +  \epsilon \frac{\w^\top}{\sqrt{||\w||_2}}\h + O(\epsilon^2)
    \end{align}
    so that
    \begin{equation}
      \nabla J(\w) = \frac{\w}{||\w||_2}.
    \end{equation}

  \end{solution}

\item $J(\w)=f(||\w||_2)$.
  \begin{solution}
    Either the chain rule or the approach with the Taylor expansion can be used
    to deal with the outer function $f$. In any case:
    \begin{align}
      \nabla J(\w) &= f'(||\w||_2) \nabla ||\w||_2 = f'(||\w||_2) \frac{\w}{||\w||_2},
    \end{align}
    where $f'$ is the derivative of the function $f$.
  \end{solution}

\item $J(\w)=f(\w^\top\a)$.

  \begin{solution}
    We have seen that $\nabla_\w \a^\top\w = \a$. Using the chain rule then yields
    \begin{align}
      \nabla J(\w)  &= f'(\w^\top\a)\nabla (\w^\top\a) \\
      =& f'(\w^\top\a) \a
    \end{align}


  \end{solution}

\end{exenumerate}

% --------------------------------------------------

\ex{Newton's method}
\label{ex:newtons_method}
Assume that in the neighbourhood of $\w_0$, a function $J(\w)$ can be described by the quadratic approximation
\begin{equation}
  f(\w)=c+ \g^\top(\w-\w_0)+\frac{1}{2}(\w-\w_0)^\top \H(\w-\w_0),
\end{equation}
where $c=J(\w_0)$, $\g$ is the gradient of $J$ with respect to $\w$, and $\H$ a
symmetric positive definite matrix (e.g.\ the Hessian matrix for $J(\w)$ at
$\w_0$ if positive definite).

\begin{exenumerate}
\item Use \exref{ex:grad-vector} to determine $\nabla f(\w)$.
  
  \begin{solution}
    We first write $f$ as
    \begin{align}
      f(\w) &= c + \g^\top(\w-\w_0)+\frac{1}{2}(\w-\w_0)^T\H(\w-\w_0) \\
            &= c - \g^\top \w_0 + \frac{1}{2} \w_0^\top\H \w_0 + \nonumber \\
            &\phantom{=} \g^\top \w + \frac{1}{2}\w^\top\H\w - \frac{1}{2} \w_0^\top \H \w - \frac{1}{2} \w^\top \H \w_0
    \end{align}
	  Using now that $\w^\top \H \w_0$ is a scalar and that $\H$ is symmetric, we have
    \begin{equation}
      \w^\top \H \w_0= (\w^\top \H \w_0)^\top = \w_0^\top \H^\top \w = \w_0^\top \H \w
    \end{equation}
    and hence
    \begin{equation}
      f(\w) = \text{const} + (\g^\top - \w_0^\top \H) \w +  \frac{1}{2}\w^\top\H\w
    \end{equation}
    With the results from \exref{ex:grad-vector} and the fact that $\H$ is symmetric, we thus obtain
    \begin{align}
      \nabla f(\w) &= \g-\H^\top \w_0 + \frac{1}{2}(\H^\top\w + \H\w) \\
                   & = \g-\H \w_0 + \H\w
    \end{align}
    
    The expansion of $f(\w)$ due to the $\w-\w_0$ terms is a bit
    tedious. It is simpler to note that gradients define a linear approximation
    of the function. We can more efficiently deal with $\w-\w_0$ by changing the
    coordinates and determine the linear approximation of $f$ as a function of
    $\v = \w-\w_0$, i.e.\ locally around the point $\w_0$. We then have
    \begin{align}
      \tilde{f}(\v) & = f(\v+\w_0)\\
                    & = c + \g^\top \v + \frac{1}{2} \v^\top \H \v
    \end{align}
    With \exref{ex:grad-vector}, the derivative is
    \begin{equation}
      \nabla_{\v} \tilde{f}(\v) = \g + \H \v
    \end{equation}
    and the linear approximation becomes
    \begin{equation}
      \tilde{f}(\v+\epsilon \h) =  c + \epsilon (\g + \H \v)^\top \h + O(\epsilon^2)
    \end{equation}
    The linear approximation for $\tilde{f}$ determines a linear approximation of $f$ around $\w_0$, i.e.\
    \begin{equation}
      f(\w + \epsilon \h) =  \tilde{f}(\w-\w_0+\epsilon \h) =  c + \epsilon (\g + \H (\w-\w_0))^\top \h + O(\epsilon^2)
    \end{equation}
    so that the derivative for $f$ is
    \begin{equation}
      \nabla_{\w}f(\w) = \g + \H (\w-\w_0) = \g - \H \w_0 + \H \w,
    \end{equation}
    which is the same result as before.
  \end{solution}
  
\item A necessary condition for $\w$ being optimal (leading either to a maximum, minimum or a saddle point) is $\nabla f(\w)=0$. 
  Determine $\w^\ast$ such that $\nabla f(\w)\big|_{\w=\w^\ast}=0$. Provide arguments why $\w^\ast$ is a minimiser of $f(\w)$.
  
  \begin{solution}
    We set the gradient to zero and solve for $\w$:
    \begin{equation}
        \g + \H(\w -\w_0) = 0 \quad \leftrightarrow \quad \w-\w_0 = - \H^{-1} \g
      \end{equation}
      so that
      \begin{equation}
        \w^\ast = \w_0 - \H^{-1} \g.
      \end{equation}
    As we assumed that $\H$ is positive definite, the inverse $\H$ exists (and
    is positive definite too).
    
    Let us consider $f$ as a function of $\v$ around $\w^\ast$, i.e.\ $\w = \w^\ast+\v$. With $\w^\ast+\v-\w_0= -\H^{-1}\g + \v$, we have
    \begin{align}
      f(\w^\ast+\v) & = c + \g^\top(- \H^{-1} \g+\v) + \frac{1}{2}(-\H^{-1}\g + \v)^\top \H (-\H^{-1}\g + \v)
    \end{align}
    Since $\H$ is positive definite, we have that $(-\H^{-1}\g + \v)^\top \H
    (-\H^{-1}\g + \v)>0$ for all $\v$. Hence, as we move away from $\w^\ast$,
    the function increases quadratically, so that $\w^\ast$ minimises $f(\w)$.
    
  \end{solution}
  
  
\item In terms of Newton's method to minimise $J(\w)$, what do $\w_0$ and $\w^\ast$ stand for?
  
  \begin{solution}
    The equation
    \begin{equation}
      \w^\ast = \w_0 - \H^{-1} \g.
    \end{equation}
    corresponds to one update step in Newton's method where $\w_0$ is the
    current value of $\w$ in the optimisation of $J(\w)$ and $\w^\ast$ is the
    updated value. In practice rather than determining the inverse $\H^{-1}$, we solve
    \begin{equation}
      \H \p = \g
    \end{equation}
    for $\p$ and then set $\w^\ast = \w_0 - \p$. The vector $\p$ is the
    search direction, and it is possible include a step-length $\alpha$ so that
    the update becomes $\w^\ast = \w_0 - \alpha \p$. The value of $\alpha$ may
    be set by hand or can be determined via line-search methods \citep[see
    e.g.][]{Nocedal1999}.
\end{solution}
\end{exenumerate}

% --------------------------------------------------

\ex{Gradient of matrix-valued functions}
\label{ex:grad-matrix}
For functions $J$ that map a matrix $\W \in \mathbb{R}^{n\times m}$ to $\mathbb{R}$, the gradient is defined as 
\begin{equation}
  \nabla J(\W) = 
  \begin{pmatrix}
    \frac{\partial J(\W)}{\partial W_{11}} & \ldots &  \frac{\partial J(\W)}{\partial W_{1m}} \\
    \vdots & \vdots & \vdots\\
    \frac{\partial J(\W)}{\partial W_{n1}} & \ldots &  \frac{\partial J(\W)}{\partial W_{nm}}
  \end{pmatrix}.
\end{equation}

Alternatively, it is defined to be the matrix $\nabla J$ such that
\begin{align}
  J(\W+\epsilon \H) &= J(\W) + \epsilon \tr(\nabla J^\top \H) + O(\epsilon^2) \label{eq:grad-matrix}\\
                    &= J(\W) + \epsilon \tr(\nabla J \H^\top) + O(\epsilon^2)
\end{align}
This definition is analogue to the one for vector-valued functions
in \eqref{eq:grad-vector}. It phrases the derivative in terms of a linear
approximation to the perturbed objective $J(\W+\epsilon \H)$ and, more formally,
$\tr \nabla J^\top$ is a linear transformation that maps
$\H \in \mathbb{R}^{n \times m}$ to $\mathbb{R}$ \citep[see e.g.][Chapter 9, for a formal treatment of derivatives]{Rudin1976}.

Let $\e^{(i)}$ be \emph{column} vector which is everywhere zero but in slot $i$
where it is 1. Moreover let $\e^{[j]}$ be a \emph{row} vector which is
everywhere zero but in slot $j$ where it is 1. The outer product
$\e^{(i)}\e^{[j]}$ is then a matrix that is everywhere zero but in row $i$ and
column $j$ where it is one. For $\H = \e^{(i)} \e^{[j]}$, we obtain
\begin{align}
  J(\W+\epsilon \e^{(i)}\e^{[j]}) &= J(\W) + \epsilon \tr((\nabla J)^\top \e^{(i)} \e^{[j]}) + O(\epsilon^2)\\
                                  &= J(\W) + \epsilon  \e^{[j]} (\nabla J)^\top \e^{(i)} + O(\epsilon^2)\\
                                  &= J(\W) + \epsilon \e^{[i]} \nabla J  \e^{(j)}+O(\epsilon^2)
\end{align}
Note that $\e^{[i]} \nabla J \e^{(j)}$ picks the element of the
matrix $\nabla J$ that is in row $i$ and column $j$, i.e.\ $\e^{[i]} \nabla J
\e^{(j)}=\partial J/\partial W_{ij}$.

Use either of the two definitions to find $\nabla
J(\W)$ for the functions below, where
$\u \in \mathbb{R}^n, \v \in \mathbb{R}^m, \A \in \mathbb{R}^{n \times m}$, and
$f: \mathbb{R} \rightarrow \mathbb{R}$ is differentiable.

\begin{exenumerate}
\item $J(\W) = \u^\top\W\v$.
  
\begin{solution}
  First method: With $J(\W) = \sum_{i=1}^n \sum_{j=1}^m u_iW_{ij}v_j$ we have
  \begin{align}
    \frac{\partial J(\W)}{W_{kl}} &= u_kv_l = (\u\v^\top)_{kl}
  \end{align}
  and hence
\begin{equation}
  \nabla J(\W) = \u\v^\top
\end{equation}

Second method: 
\begin{align}
  J(\W + \epsilon \H) & = \u^\top (\W+\epsilon \H)\v\\
                      & = J(\W) + \epsilon \u^\top \H \v\\
                      & = J(\W) + \epsilon \tr(\u^\top \H \v)\\
                      & = J(\W) + \epsilon \tr(\v \u^\top \H)
\end{align}
Hence:
\begin{align}
  \nabla J(\W) & = \u\v^\top
\end{align}

\end{solution}

\item $J(\W) = \u^\top(\W+\A)\v$.
\begin{solution}
  Expanding the objective function gives $J(\W) = \u^\top \W \v
  + \u^\top \A \v$. The second term does not depend on $\W$. With the previous
  question, the derivative thus is
  \begin{equation}
    \nabla J(\W) = \u \v^\top
  \end{equation}
  
\end{solution}
\item $J(\W) = \sum_n f(\w_n^\top\v)$, where $\w_n^\top$ are the rows of the matrix $\W$.
  
  \begin{solution}
    First method:
    \begin{align}
      \frac{\partial J(\W)}{\partial W_{ij}} &= \sum_{k=1}^n \frac{\partial}{\partial W_{ij}} f(\w_k^\top\v) \\
                                             &= f'(\w_i^\top\v) \frac{\partial}{\partial W_{ij}} \underbrace{\w_i^\top\v}_\text{$\sum_{j=1}^m W_{ij}v_j$} \\
                                             &= f'(\w_i^\top\v)v_j
    \end{align}
    Hence
    \begin{equation}
      \nabla J(\W) = f'(\W\v)\v^\top,
    \end{equation}
    where $f'$ operates element-wise on the vector $\W\v$.
    
    Second method:
    \begin{align}
      J(\W) &= \sum_{k=1}^n f(\w_k^\top\v)\\
            &= \sum_{k=1}^n f(\e^{[k]}\W\v),
    \end{align}
    where $\e^{[k]}$ is the unit row vector that is zero everywhere but for element $k$ which equals one. We now perform a perturbation of $\W$ by $\epsilon \H$.
    \begin{align}
J(\W + \epsilon \H) &= \sum_{k=1}^n f(\e^{[k]}(\W + \epsilon \H)\v) \\
                    &= \sum_{k=1}^n f(\e^{[k]}\W\v + \epsilon\e^{[k]}\H\v) \\
         					  &= \sum_{k=1}^n (f(\e^{[k]}\W\v) + \epsilon f'(\e^{[k]}\W\v)\e^{[k]}\H\v + O(\epsilon^2)\\
                    &= J(\W) + \epsilon \left( \sum_{k=1}^n  f'(\e^{[k]}\W\v)\e^{[k]}\right) \H \v + O(\epsilon^2)
    \end{align}
    The term $f'(\e^{[k]}\W\v) \e^{[k]}$ is a row vector that equals $(0, \ldots, 0,f'(\e^{[k]}\W\v), 0, \ldots, 0)$. Hence, we have
    \begin{equation}
      \sum_{k=1}^n  f'(\e^{[k]}\W\v)\e^{[k]}) = f'(\W\v)^\top
    \end{equation}
    where $f'$ operates element-wise on the column vector $\W\v$. The perturbed objective function thus is
    \begin{align}
      J(\W + \epsilon \H) & = J(\W) + \epsilon f'(\W\v)^\top \H \v  + O(\epsilon^2)\\
                          & =  J(\W) + \epsilon \tr\left(f'(\W\v)^\top \H \v\right)  + O(\epsilon^2)\\
                          & =  J(\W) + \epsilon \tr\left(\v f'(\W\v)^\top \H \right)  + O(\epsilon^2)
    \end{align}
    Hence, the gradient is the transpose of $\v f'(\W\v)^\top$, i.e.\
    \begin{equation}
      \nabla J(\W) = f'(\W\v) \v^\top
\end{equation}
\end{solution}

\item $J(\W) = \u^\top \W^{-1}\v$ (\emph{Hint: $(\W+\epsilon \H)^{-1} = \W^{-1} -\epsilon \W^{-1}\H\W^{-1}+O(\epsilon^2)$.})
  \begin{solution}
    We first verify the hint:
    \begin{align}
      \left(\W^{-1} -\epsilon \W^{-1}\H\W^{-1}+O(\epsilon^2)\right) (\W+\epsilon \H)&=  \I + \epsilon \W^{-1} \H - \epsilon \W^{-1}\H + O(\epsilon^2)\\
                                                                                    &= \I + O(\epsilon^2)
    \end{align}
    Hence the identity holds up to terms smaller than $\epsilon^2$, which is
    sufficient we do not care about terms of order $\epsilon^2$ and smaller 
    in the definition of the gradient in \eqref{eq:grad-matrix}.
    
    Let us thus make a first-order approximation of the perturbed objective $J(\W + \epsilon \H)$: 
    \begin{align}
      J(\W + \epsilon\H) &= \u^\top(\W + \epsilon\H)^{-1}\v \\
                         &\overset{\textrm{hint}}{=} \u^\top(\W^{-1} - \epsilon \W^{-1}\H \W^{-1} + O(\epsilon^2))\v \\
                         &= \u^\top \W^{-1}\v - \epsilon \u^\top\W^{-1}\H\W^{-1}\v + O(\epsilon^2) \\
                         &= J(\W) -  \epsilon \tr\left(\u^\top\W^{- 1}\H\W^{-1}\v\right) + O(\epsilon^2) \\
                         &= J(\W) - \epsilon \tr\left(\W^{-1}\v\u^\top\W^{-1}\H\right) + O(\epsilon^2)
    \end{align}
    Comparison with \eqref{eq:grad-matrix} gives
    \begin{equation}
      \nabla J^\top = -\W^{-1}\v\u^\top\W^{-1}
    \end{equation}
    and hence
    \begin{equation}
      \nabla J = -\W^{-\top}\u\v^\top\W^{-\top},
    \end{equation}
where $\W^{-\top}$ is the transpose of the inverse of $\W$.

\end{solution}
\end{exenumerate}

% --------------------------------------------------

\ex{Gradient of the log-determinant}
\label{ex:grad-log-det}
The goal of this exercise is to determine the gradient of
\begin{equation}
  J(\W) = \log | \det(\W)|.
\end{equation}

\begin{exenumerate}
\item Show that the $n$-th eigenvalue $\lambda_n$ can be written as
  \begin{equation}
    \lambda_n=\v_n^\top \W \u_n,
  \end{equation}
  where $\u_n$ is the $n$th eigenvector and $\v_n$ the $n$th column vector of $\U^{-1}$, with $\U$ being the matrix with the eigenvectors $\u_n$ as columns.
  
  \begin{solution}
    As in \exref{ex:eigenvalue-decomposition}, let $\U\Lambdab \V^\top$ be the
    eigenvalue decomposition of $\W$ (with $\V^\top = \U^{-1}$). Then $\Lambdab = \V^\top
    \W\U$ and
    \begin{align}
      \lambda_n &= \e^{[n]}\Lambdab\e^{(n)} \\
                &= \e^{[n]}\V^\top \W \U\e^{(n)} \\
                &= (\V\e^{(n)})^\top \W\U\e^{(n)} \\
                &= \v_n^\top \W\u_n,
    \end{align}
    where $\e^{(n)}$ is the standard basis (unit) vector with a 1 in the $n$-th
    slot and zeros elsewhere, and $\e^{[n]}$ is the corresponding row vector.
  \end{solution}
  
\item Calculate the gradient of $\lambda_n$ with respect to $\W$, i.e. $\nabla \lambda_n(\W)$.

  \begin{solution}
    With \exref{ex:grad-matrix}, we have
    \begin{align}
      \nabla_{\W} \lambda_n(\W) &= \nabla_{\W} \v_n^\top \W \u_n = \v_n \u_n^\top.
    \end{align}
  \end{solution}
  
\item Write $J(\W)$ in terms of the eigenvalues $\lambda_n$ and calculate $\nabla J(\mathbf{\W})$.
  \begin{solution}
    In \exref{ex:trace-determinants-eigenvalues}, we have shown that $\det(\W) = \prod_i \lambda_i$ and hence $|\textrm{det}(W)| = \prod_i |\lambda_i|.$
    \begin{enumerate}
    \item[(i)] If $\W$ is positive definite, its eigenvalues are positive and
      we can drop the absolute values so that $|\det(W)| = \prod_i \lambda_i$.
    \item[(ii)] If $\W$ is a matrix with real entries, then $\W\u = \lambda\u$
        implies $\W\bar{\u} = \bar{\lambda}\bar{\u}$, i.e. if $\lambda$ is a
        complex eigenvalue, then $\bar{\lambda}$ (the complex conjugate of
        $\lambda$) is also an eigenvalue. Since $|\lambda|^2 =
        \lambda\bar{\lambda}$,
        \begin{align}
          |\det(\W)| = \left(\prod_{\lambda_i \in \mathbb{C}} \lambda_i \right) \left(\prod_{\lambda_j \in \mathbb{R}} |\lambda_j| \right).
        \end{align}
      \end{enumerate}
      Now we can write $J(\W)$ in terms of the eigenvalues:
      \begin{align}
        J(\W)  &= \log |\det(\W)| \\
               &= \log\left(\prod_{\lambda_i \in \mathbb{C}} \lambda_i \right) \left(\prod_{\lambda_j \in \mathbb{R}} |\lambda_j|\right) \\
               &= \log\left(\prod_{\lambda_i \in \mathbb{C}} \lambda_i \right) + \log\left(\prod_{\lambda_j \in \mathbb{R}} |\lambda_j| \right) \\
               &= \sum_{\lambda_i \in \mathbb{C}} \log \lambda_i + \sum_{\lambda_j \in \mathbb{R}} \log |\lambda_j|.
      \end{align}
      Assume that the real-valued $\lambda_j$ are non-zero so that
      \begin{align}
        \nabla_{\W} \log |\lambda_j| & = \frac{1}{|\lambda_j|} \nabla_{\W} |\lambda_j|\\
                                     & =  \frac{1}{|\lambda_j|} \textrm{sign}(\lambda_j)\nabla_{\W}\lambda_j
      \end{align}
      Hence
      \begin{align}
        \nabla J(\W) &= \sum_{\lambda_i \in \mathbb{C}}  \nabla_{\W}\log \lambda_i + \sum_{\lambda_j \in \mathbb{R}}  \nabla_{\W} \log |\lambda_j|\\ 
                     &= \sum_{\lambda_i \in \mathbb{C}} \frac{1}{\lambda_i} \nabla_{\W} \lambda_i + 
                       \sum_{\lambda_i \in \mathbb{R}} \frac{1}{|\lambda_i|} \textrm{sign}(\lambda_i)\nabla_{\W}\lambda_i\\
                     &= \sum_{\lambda_i \in \mathbb{C}} \frac{\v_i\u_i^\top}{\lambda_i} + 
                       \sum_{\lambda_i \in \mathbb{R}} \frac{\textrm{sign}(\lambda_i)\v_i\u_i^\top}{|\lambda_i|} \\
                     &= \sum_{\lambda_i \in \mathbb{C}} \frac{\v_i\u_i^\top}{\lambda_i} + 
                       \sum_{\lambda_i \in \mathbb{R}} \frac{\v_i\u_i^\top}{\lambda_i} \\
                     &= \sum_i  \frac{\v_i\u_i^\top}{\lambda_i}.
      \end{align}
      
    \end{solution}
  \item Show that
    \begin{equation}
      \nabla J(\W) = (\W^{-1})^\top.
    \end{equation}
    
    \begin{solution}
    This follows from \exref{ex:eigenvalue-decomposition} where we have found
    that
    \begin{equation}
      \W^{-1}= \sum_i \frac{1}{\lambda_i}\u_i \v_i^\top.
    \end{equation}
    Indeed:
    \begin{align}
      \nabla J(\W) = \sum_i  \frac{\v_i\u_i^\top}{\lambda_i} = \sum_i \frac{1}{\lambda_i} (\u_i\v_i^\top)^\top =  \left(\W^{-1}\right)^\top.
    \end{align}
    
  \end{solution}
\end{exenumerate}

% --------------------------------------------------

\ex{Descent directions for matrix-valued functions}
\label{ex:descent-directions-for-matrix-valued-functions}
Assume we would like to minimise a matrix valued function $J(\W)$ by gradient
descent, i.e. the update equation is
\begin{equation}
  \W \leftarrow \W - \epsilon \nabla J(\W),
\end{equation}
where $\epsilon$ is the step-length. The gradient $\nabla J(\W)$ was defined in
\exref{ex:grad-matrix}. It was there pointed out that the gradient defines a
first order approximation to the perturbed objective function $J(\W+\epsilon
\H)$. With \eqref{eq:grad-matrix},
\begin{align}
  J(\W - \epsilon \nabla J(\W)) & = J(\W) - \epsilon \tr(\nabla J(\W)^\top \nabla J(\W)) + O(\epsilon^2)
\end{align}
For any (nonzero) matrix $\M$, it holds that
\begin{align}
  \tr(\M^\top \M) &= \sum_{i} (\M^\top\M)_{ii}\\
                  &= \sum_i \sum_j (\M^\top)_{ij} (\M)_{ji}\\
                  &= \sum_i \sum_j M_{ji} M_{ji}\\
                  &= \sum_{ij} (M_{ji})^2\\
                  &> 0, \label{eq:trace-nonnegativity}
\end{align}
which means that $\tr(\nabla J(\W)^\top \nabla J(\W)) > 0$ if the gradient is nonzero, and hence
\begin{equation}
  J(\W - \epsilon \nabla J(\W)) < J(\W)
\end{equation}
for small enough $\epsilon$. Consequently, $\nabla J(\W)$ is a descent
direction. Show that $\A^\top \A \nabla J(\W)\B\B^\top$ for non-zero matrices
$\A$ and $\B$ is also a descent direction or leaves the leaves the objective invariant.

\begin{solution}
  As in the introduction to the question, we appeal to \eqref{eq:grad-matrix} to
  obtain
  \begin{align}
    J(\W - \epsilon \nabla J(\W)\A^\top\A \B\B^\top) & = J(\W) - \epsilon \tr(\nabla J(\W)^\top \A^\top \A \nabla J(\W)\B\B^\top) + O(\epsilon^2)\\
                                                     & =  J(\W) - \epsilon \tr(\B^\top \nabla J(\W)^\top \A^\top \A \nabla J(\W)\B) + O(\epsilon^2),
  \end{align}
  where $\tr(\B^\top \nabla J(\W)^\top\A^\top \A \nabla J(\W)\B)$ takes the form
  $\tr(\M^\top \M)$ with $\M= \A \nabla J(\W)\B$. With
  \eqref{eq:trace-nonnegativity}, we thus have $\tr(\B^\top \nabla J(\W)^\top \A^\top
  \A \nabla J(\W)\B) > 0$ if $\A \nabla J(\W)\B$ is non-zero, and hence
  \begin{equation}
    J(\W - \epsilon \A^\top \A \nabla J(\W) \B\B^\top) < J(\W)
  \end{equation}
  for small enough $\epsilon$. We have equality if $\A \nabla J(\W)\B = 0$,
  e.g.\ if the columns of $\B$ are all in the null space of $\nabla J$.
\end{solution}

% Possible future question
%\item What conditions must $\A$ and $\B$ satisfy for $\A^\top \A \nabla
%  J(\W)\B\B^\top$ to be a descent direction?





